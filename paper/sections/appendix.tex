%\section{GW Event sampling} \label{sec:MCMC}
%To generate simulated gravitational wave data, we first develop an estimate of the probability distribution. This is a very low dimensional probability distribution, with only three variables of interest, luminosity distance, luminosity distance errors and sky location bounds. The training data was taken from the third round of LIGO observations\cite{GWTC_2,GWTC_3}. After the candidate selection is performed, this contains only 76 data points.

%Since this problem is so low dimensional and we have so few data points, we can fairly easily produce a Voronoi tesselation of all observations. The Voronoi tesselation of $N$ is a partition of space with $N$ regions such that each of the $N$ points is contained in exactly one region and all points within the region corresponding to point $i$ are closer to point $i$ than any other point. As such, the reciprocal of the volume of each cell gives an estimate on the density of points within a given region.

%To describe our procedure formally, let $D=\{\vec{x}_i : 1\leq i\leq N\}$ be the set of training points and $\{V_i\}$ be the corresponding set of cells in the Voronoi diagram. Let $NN(\vec{y}):\mathbb{R}^d\to \{i\in\mathbb{N} : 1\leq i\leq N\}$ be the function that maps the point $\vec{y}$ to the index $i$ of its nearest neighbor in the set $D$, and let $v(V_i)$ be the volume of the $i^{th}$ cell. We then assign a probability distribution to each point $\vec{y}\in\mathbb{R}^d$, 

%\begin{equation}
%p(\vec{y} | D) = \left( v( V_{NN(\vec{y})} ) \right)^{-1}
%\end{equation}

%With this probability distribution in place, we can use Markov Chain Monte Carlo to generate event samples.

%It was necessary to rescale the sky location uncertainty by a factor of $1000$ to avoid numeric instabilities before producing tesselations. Training data and MCMC samples are shown in figure \ref{fig:MCMC}

\section{\label{sec:clust_gen} Simulated galaxy catalogs}
After producing a simulated event, we need to produce a catalog of galaxies within the credible interval of sky locations and radial distances obtained from the GW event sample. The volumes produced by typical events are very large, on the order of $1$ \si{Gpc^3}. Volumes this large will exhibit large scale structures. To simplify our analysis we only consider clusters of galaxies and not larger structures such as superclusters or filaments.

To estimate the density of galaxy clusters we use a survey by S. Hansen et al. of a $395$ \si{deg^2} region of sky observed in the SDSS catalog \cite{Hansen_2005}. This analysis found 12830 clusters with redshifts in the range $0.07\leq z\leq 0.3$, for which this catalog is considered complete. Using a value of $H_0=70$ \si{km.s^{-1}.Mpc^{-1}}, we arrive at a very crude estimate of cluster density of $1.5\times 10^{-4}$ \si{Mpc^{-3}}.

Assuming that the mass of normal matter in a cluster follows the Press-Schechter function \cite{Press_1974} and that the number of galaxies is proportional to mass we can express the probability distribution that a cluster has $n$ galaxies as

\begin{equation}
p(n | n_c, \alpha) \propto \left( \frac{n}{n_c} \right)^{\alpha}e^{n/n_c}
\end{equation}

where $n_c$ and $\alpha$ are paramters that should be fitted to the data. Using the SDSS catalog we obtain estimates $\alpha = 7.695$ and $n_c = 3$. The authors also find that the characteristic radius for a galaxy cluster goes as $r_c ~ \mathcal{N}_{gals}^\beta$ for $\beta=0.56$ \cite{Hansen_2005}. We take the offset of any galaxy from the center of mass to follows a three-dimensional Gaussian distribution with $\sigma_r = r_c$. Under the assumption that the virial theorem is obeyed and that velocities follow a normal distribution, we can derive an expression for $\sigma_v$ in terms of $r_c$:

\begin{widetext}
\begin{align}
    \left<U\right> &= \frac{4\pi G\mu}{\sqrt{2\pi r_c^2}} \int r e^{ -r^2 / 2r_c^2} dr = \sqrt{8\pi m}r_c = 2\pi m\sigma_v^2 = \left<K\right> \nonumber\\
    &\implies \sigma_v^2 = \frac{G r_c}{\sqrt{\pi/2}}\frac{\mu}{m}.
\end{align}
\end{widetext}
Here we have taken $\mu$ to be the reduced mass of the galaxy and the center of gravity of the cluster. It is well known that galaxy clusters tend to be dominated by dark matter \cite{Clowe_2006, Hoh_2020}. Therefore, in all of our numerical work we use the approximation $\mu / m = 1$.

This model for catalog generation makes a number of unrealistic assumptions. We have treated cluster densities as uniform within the range of interest, but we should expect this to have a redshift dependence. We have also ignored the role of dark matter which should constitute the majority of a cluster's mass. It may be interesting to account for these affects in future work, but the model presented here is sufficient for a toy model. However, this model does produce correlations between the peculiar motions of galaxies and their locations, which is a useful test of the robustness of our estimation of Hubble's constant.

Now that we have described the details of our cluster generation procedure, we need to address one more important detail in our cluster generation. In order to make our data realistic, there should be a true host galaxy. We expect this galaxy to be near the center of our volume credible interval. To account for this, we select a cluster at random to contain the host where the probability that any given cluster contains the host is proportional to the number of galaxies it contains. After selecting this cluster, we move it to the center of the GW localization region and apply a Gaussian displacement.

To get a crude estimate of the width of this Gaussian displacement, we start by considering a sphere which has an equivalent volume. Assuming that this volume represents a $95\%$ confidence interval, the radius of this galaxy should be $2\sigma$. Thus, we select a Gaussian width on the host cluster displacement,

\begin{equation}
  \sigma = \frac{1}{2}\left(\frac{3}{4\pi} V\right)^{1/3}.
\end{equation}

\section{\label{sec:code} Data and Code}

The code for both galaxy cluster simulation and the calculation of the posterior distribution of $H_0$, as well as the data products for each of these, can be found at \url{https://github.com/REDACTED/phys_403_final_project}.
